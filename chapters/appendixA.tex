\chapter{可行性分析结果}

\section{法律可行性}

本云盘并未使用付费的商用船舰,依赖软件均为开源社区软件,且不进行公开发售,不从中获利。

开发过程与使用遵守云盘依赖软件的GPL等协议,不侵犯知识产权

\section{开发可行性}

本云盘所依赖的工具、技术、第三方软件都是已有大规模应用的,十分成熟的软件

本云盘没有算法上的技术限制,主要均为业务代码

本云盘的开发团队对所需技术较为了解,上手迅速

\section{部署可行性}

本云盘将提供可在Debian Linux系统上的安装、运行脚本,但使用者需要与服务器处在同一局域网,或者服务器需要有公网ip。

本云盘对主机性能要求不高

\section{推广可能性}

现有的云盘系统虽然较为成熟,但操作繁琐,功能限制多:必须安装客户端,解压收费,下载限速等。本云盘系统集成各项方便用户使用的功能的同时做到轻量化,只用浏览器即可使用完整功能,因此具有切实可行性。
