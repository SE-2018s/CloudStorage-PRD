\chapter{具体需求}
\section{功能需求}

本章节描述xyz云盘系统所必须执行的基本动作,以及其输入、输出,以及中间处理的过程。
\subsection{R.XYZ.CLOUDSTORAGE.USER.LOGIN.001 用户:显示初始登陆界面 }

\subsubsection{介绍}
打开客户端之后,用户需要登陆,才可以访问自己的存储数据。登陆需要友好的登录界面。

\subsubsection{输入}
用户打开客户端。

\subsubsection{处理}
生成登录窗口,包括用户名密码等窗口,其中,密码需要密码显示的保护处理。
\subsubsection{输出}
显示友好的登录窗口。

\subsection{R.XYZ.CLOUDSTORAGE.USER.LOGIN.002 用户:密码验证 }

\subsubsection{介绍}
在看到登陆界面之后,用户需要输入用户名密码,并通过”登陆“键提交密码,以进行验证并登陆。

\subsubsection{输入}
用户输入用户名和密码,按”登陆“键,或者”Enter“键。

\subsubsection{处理}
客户端将用户名密码,结合时间戳进行密码学处理之后打包,发送给服务器端。
服务器端通过密码学手段验证密码。如果密码正确,返回带有时间戳的cookie给客户端。如果密码错误,则返回错误信息给客户端。错误次数不能超过4次。超过则禁止ip尝试同一用户。如果发生通信异常,则保存异常信息到异常日志中,同时客户端重传,直到超时。

\subsubsection{输出}
如果密码正确,显示用户的根文件夹;如果密码错误,显示”密码错误“,以及错误次数,进行警告。如果通信异常,则显示通信异常信息。

\subsection{R.XYZ.CLOUDSTORAGE.USER.LOGOUT.001 用户:退出 }

\subsubsection{介绍}
用户能够退出客户端,结束本次使用。

\subsubsection{输入}
用户按下“退出”按钮,或者直接关闭客户端。

\subsubsection{处理}
用户“按下退出”按钮后,客户端发包给服务器端,提示结束本次使用。服务器端将该用户设置为“离线”状态。
用户通过其他方式直接关闭客户端时(包括强制关闭,关机等),服务器端定期检查客户端是否在线,如果超时无回应,则自动设置为“离线”状态。

\subsubsection{输出}
客户端被关闭。

\subsection{R.XYZ.CLOUDSTORAGE.FILE.BASIC.001 文件:打开文件夹}

\subsubsection{介绍}
用户可以打开文件夹,查看所包含的文件及子文件夹。

\subsubsection{输入}
用户双击文件夹,或者选中文件夹之后点击打开选项。

\subsubsection{处理}
客户端将打开文件夹的指令打包,发送到服务器,服务器传回文件夹内容。客户端接收之后,切换路径到所选中的文件夹,并展示其所包含的文件及文件夹。

异常:如果通信失败,则客户端重传,直到超时。

\subsubsection{输出}
若无异常发生,用户可以看到文件夹中的文件以及子文件夹。

若有异常发生,则将异常信息保存到日志中,并显示友好界面提示异常。

\subsection{R.XYZ.CLOUDSTORAGE.FILE.BASIC.002 文件:查看文件属性}

\subsubsection{介绍} 
用户可以查看文件属性,包括文件大小、创建时间、修改时间、上次下载时间、上次访问时间、用户权限、md5值等。

\subsubsection{输入} 
用户选中文件,再点击属性选项。

\subsubsection{处理}
客户端将该请求发送到服务器端,服务器查询属性,并传回客户端。客户端显示属性值。
异常:如果通信失败,则客户端重传,直到超时。
若有异常发生,则将异常信息保存到日志中,并显示友好界面提示异常。

\subsubsection{输出}
客户端接收到之后弹窗显示各项属性值。


\subsection{R.XYZ.CLOUDSTORAGE.FILE.BASIC.003 文件:排序显示}

\subsubsection{介绍} 
用户可以按照文件名以及属性值选择排序显示的方式。

\subsubsection{输入} 
用户点击“排序”选项,并选中所要排序的依据。

\subsubsection{处理}
客户端按照用户的选择,调用排序函数,排好序之后进行显示。

\subsubsection{输出}
客户端按照排序结果显示文件和文件名。



\subsection{R.XYZ.CLOUDSTORAGE.FILE.BASIC.004 文件:重命名}

\subsubsection{介绍} 
用户可以对文件以及文件夹进行重命名操作。

\subsubsection{输入} 
用户选中文件或文件名,右键,选中“重命名”选项。

\subsubsection{处理}
客户端将重命名的原名字和新名字打包发给服务器端,服务器检查该用户是否有重命名权限,如果有则常识修改,如果明明合法不冲突则返回“成功”的信息,否则,返回“失败”的信息。

异常:如果通信失败,则客户端重传,直到超时。
若有异常发生,则将异常信息保存到日志中,并显示友好界面提示异常。

\subsubsection{输出}
如果成功,则刷新当前目录,显示最新的名字。如果失败,则提示失败。


\subsection{R.XYZ.CLOUDSTORAGE.FILE.BASIC.005 文件:复制、粘贴、剪切}

\subsubsection{介绍} 
用户可以对部分文件以及文件夹进行复制、粘贴、剪切等操作。

\subsubsection{输入} 
复制:用户选中需要操作的文件和文件夹,右键,选中“复制”选项。

剪切:用户选中需要操作的文件和文件夹,右键,选中“剪切”选项。

粘贴:用户在所要粘贴的文件夹中,右键,选中“粘贴”选项。需要注意的是,必须有之前“复制”或“剪切”的操作记录,“粘贴”选项才可选。

\subsubsection{处理}
复制:客户端将用户选中要复制的项的完全名字(包括路径)存储到cache中。

剪切:客户端将用户选中要剪切的项的完全名字(包括路径)存储到cache中。

粘贴:客户端将执行粘贴的文件夹的路径,以及之前复制或者剪切的类型一起打包,发给服务器端。服务器接收之后检查用户对目标文件夹是有写入权限,对源文件是否有读权限,命名是否冲突。如果没有就返回“没有权限”信息;如果有,如果是复制,则复制到目标文件夹,如果是剪切,则先复制,再删除(删除之前还要检查源文件的写权限)。最后返回“成功”给客户端。

\subsubsection{输出}
如果粘贴成功,则刷新当前文件夹,显示最新结果。
如果粘贴失败,则弹窗提示粘贴失败。

\subsection{R.XYZ.CLOUDSTORAGE.FILE.HIGH.001 文件:回收站}

\subsubsection{介绍} 
用户可以将文件以及文件夹移入回收站,回收站在7天后自动删除,用户也可以在回收站住主动删除。用户也可以讲文件以及文件夹从回收站中移出。

\subsubsection{输入} 
移入回收站:用户选中文件或文件夹,右键,选中“移入回收站”选项。

移出回收站:用户在回收站中选中文件或文件夹,右键,选中“移出回收站”选项。

彻底删除:用户在回收站中选中文件或文件夹,右键,选中“彻底删除”。

\subsubsection{处理} 
移入回收站:客户端将该文件或文件夹名字打包发送到服务器端。服务器端检查该用户是否有权限删除该文件或文件夹。如果有则移入“回收站”中并返回“成功”,否则返回“失败”。客户端收到信息后,如果成功,则刷新当前文件夹;如果失败,则弹窗提示。

移出回收站:客户端将该文件或文件夹名字打包发送到服务器端。服务器端检查该文件或文件夹复原之后是否有命名冲突等。如果无冲突则移出“回收站”中并返回“成功”,否则返回“失败”。客户端收到信息后,如果成功,则刷新回收站;如果失败,则弹窗提示。

彻底删除:客户端将该文件或文件夹名字打包发送到服务器端。服务器从回收站中删除。返回“成功”。客户端收到信息后,如果成功,则刷新回收站;如果失败,则弹窗提示。

异常:如果通信失败,则客户端重传,直到超时。
若有异常发生,则将异常信息保存到日志中,并显示友好界面提示异常。

\subsubsection{输出} 
移入回收站:如果成功,则刷新当前文件夹,否则弹窗提醒。

移出回收站:如果成功,则刷新回收站,否则弹窗提醒。

彻底删除:如果成功,则刷新当前文件夹,否则弹窗提醒。

\subsection{R.XYZ.CLOUDSTORAGE.FILE.HIGH.002 文件:收藏夹}

\subsubsection{介绍} 
用户可以将文件以及文件夹加入收藏夹,以便快速访问。

\subsubsection{输入} 
加入收藏夹:用户选中文件或文件名,右键,选中“加入收藏夹”选项。
移出收藏家:用户在收藏夹中选中文件或文件名,右键,选中“移出收藏夹”选项。

\subsubsection{处理}
加入收藏夹:客户端将用户选中的文件名打包发给服务器端,服务器将该文件或文件夹加入所维护的收藏夹数据结构中。返回成功。
移出收藏夹:客户端将用户选中的文件名打包发给服务器端,服务器将该文件或文件夹从所维护的收藏夹数据结构中删除。返回成功。
异常:如果通信失败,则客户端重传,直到超时。
若有异常发生,则将异常信息保存到日志中,并显示友好界面提示异常。

\subsubsection{输出}
如果成功,并且是当前目录是收藏夹,则刷新当前目录,显示最新的名字。如果失败,则提示失败。

















逐条列出与本特性相关的功能需求。包括项目如何响应预期的错误输入,非法条件和无效输入。需求应该简明,完整,不含糊,可验证,必要的。 当需要的信息不确定的时候使用“待定”。
\subsubsection{输入}
This section consists of:
A. Detailed description of all input data of the function, including:
Source of input
Quantify
Measurement units
Timing requirements
Valid input range that contains the precision and tolerance
B. Reference of interface specification or interface control document that are provided in proper place.

本子段落应包含下列内容:

A. 对该功能所有输入数据的详细描述,包括:

		输入来源
		数量
		度量单位
		时间要求
		包含精度和容忍度的有效输入范围
		
B. 在适当的地方提供的对接口规格或接口控制文档的参考。
\subsubsection{处理}
Describes all the operations on the input data, and the process to get the output data, including the following specifications:
A. Verification of input data
B. Exact order of the operations, including the time sequene of each event.
C. Response to exception, such as:
		 Overflow
		Communication failure
		Error process
D. Any method used to transfer the input data to the output data. (such as equation,mathematic algorithm and logical operation)
For example.
		The formula to calculate the income tax in a pay roll.
		the weather model used for weather forecast
E. Verification of output data.
本子段落应描述对输入数据所执行的所有操作和如何获得输出的过程。这包括下列规格:

A. 输入数据的有效性检测。

B. 操作的确切次序,包括各事件的时序。

C. 对异常情况的回应,例如:
		溢出
		通信失败
		错误处理
D. 用于把系统输入转换到相应输出的任何方法(诸如方程式,数学算法,逻辑操作)。例如,这可能描述下列方面:
		对工资单里代扣所得税的计算公式。
		用于气象预报的气象模型。
		
E.	对输出数据的有效性检测。
\subsubsection{输出}
This section should include:
A. The detailed description of output data of the function, including:
		Target to output to (Such as a printer or a file)
		Quantity
		Measurement units
		Time sequence
		Valid output range including the precision and tolerance
		Process of the invalid value.
		Error message.
B. Reference of interface specification or interface control document that are provided in proper place.
For the systems with their requirements focused on the input/output actions, all the important input/output actions and the time sequences of the input/output pairs should be described in the SRS. In a system that inputs and actions are memorized as the basis for the reactions to be taken, the timing sequence for the input/output pairs must be available here. This kind of functional action is similar to a status machine.  
本子段落应包含:

A. 对该功能所有输出数据的详细描述,这个描述包括:
		输出的到何处(如打印机,文件)
		数量
		度量单位
		时序
		包含精确度和容忍度的有效输出范围
		对非法值的处理
		错误消息
		
B. 在适当的地方提供对接口规格或接口控制文档的参考。

此外,对那些需求集中在输入/输出行为的系统,SRS应描述所有重要的输入/输出行为及输入输出对的次序。对一个需要记忆其行为以根据输入和过去的行为进行反应的系统,输入输出对的次序是要求的;这种功能行为就类似于有限状态机。
\section{性能需求}
<If there are performance requirements, state them here and explain their rationale, to help the developers understand the intent and make suitable design choices. Specifies the timing relationships for real time systems. Such requirements should be made as specific as possible. >

如果有性能方面的需求,在这里列出并解释他们的原理。以帮助开发者理解意图以做出正确的设计选择。在实时系统中的时序关系。保证需求尽可能的详细而精确。
\subsection{性能需求1}
Describes the statically and dynamically quantized requirements on the software (or the interaction between user and the software)
Static quantized requirement could include:
A. Maximum number of terminal supported.
B. Maximum number of users that can use the software at the same time.
C. Maximum number of files and records to be processed
D. Maximum size of  tables and files
Dynamically quantized requirements could include:
A. Specific duration of normal value and peak value of workload (e.g., one hour)
B. Number of event and task and data volume to be processed 
All these requirements should be described by measurable term, for example, saying "95% of the events should be processed in 1 second", instead of saying "the operator need not wait for the business to complete."
Note: The quantized constraint of a detailed requirement should be described in the subsection of the detailed requirement.
本子章节应从整体上描述静态和动态的量化的对软件(或人与软件交互)的需求。

静态的量化需求可能包括:

A. 支持的终端数目。

B. 支持的同时使用的用户数目。

C.处理的文件和记录的数目。

D.表和文件的大小。

动态的量化需求可能包括:

A. 在正常和峰值工作量条件下特定时间段(如一小时)

B. 处理的事务和任务的数目以及数据量。

所有的这些需求应以可测量的术语进行描述,例如所有的操作应在1秒内被处理完成,而不是描述成操作员不必等待操作的完成。

注意: 用于一个具体功能的量化限制通常在该功能的处理子章节中描述。
\section{外部接口需求}
\subsection{用户接口}
<The interface of the system with the User and vice versa should be explained in detail. >

详细描述系统与用户之间的接口

This section should include:
A. Features that must be supported by the software for eachman-machine interface. For example, if the user operates from a display terminal, then the following should be included:
		Screen format required
		Page layout and content of report and menu
		Timing sequence for input and output
		Usage of some functional key combinations
B. Every aspect about the use of the system's user interface. It could be a list that shows the user what should do and what should not do.  For example, an option of overlong or overshort message. . And same as other requirements, these requirements should be easily verified. For example, saying "A level 4 typist can finish function X in Z minutes after a one-hour training." instead of "A typist can finish function X"	

这应描述下述内容:

A. 对每种人机界面,软件所必须支持的特性。例如,如果系统用户通过一个显示终端进行操作,那么应包含下述内容:
要求的屏幕格式
页面规划及报告或菜单的内容
输入和输出的相关时序
一些组合功能键的用法

B. 与系统用户接口使用相关的所有方面。这可能只是一个简单的关于系统怎样展示给用户而该做什么和不该做什么的列表。例如提供关于长或短错误消息选项。和所有其它需求一样,这些需求也应能被检验,例如,四级打字员经一小时的培训后能在Z分钟内完成功能X,而不是一个打字员能完成功能X。

\subsection{软件接口}
<The interface with other system/modules/projects should be explained in detail. >

详细描述与其他系统 /模块 /项目之间的接口

Describes how to use the other (required) software products. (such as data management system, operation system, or algorithm tools package), and the interfaces to other application systems (such as interfaces between the protocol process system and the database management system )
For each required software product, following information should be provided:
A. Name
B. Mnemonic symbol
C. Version number
D. Source
For each interface, this section should:
A. Discuss the objective of the required software.
B. Define the interfaces by content and format of message/function. If the interfaces have been clearly described in other documents, it is not necessary to describe in detail here. But the reference of those documents should be given.

在此应描述如何使用其它(必需的)软件产品(例如,数据管理系统,操作系统,或算法工具包),以及与其它应用系统的接口(例如,协议处理系统和数据库管理系统之间的接口)。

对每个必需的软件产品,应提供下列信息:
A.	名字
B.	助记符
C.	版本号
D.	来源

对每个接口,本部分应:

A .	讨论与本软件产品相关的接口软件的目的。

B.	按消息/函数内容和格式定义接口。如果接口已在其它文档中很清楚地描述,就没有必要在这儿进行详细描述,但需说明应参考的文档。

\subsection{硬件接口}
<The interface with other hardware components should be explained in detail. >

详细描述与硬件的接口

Describes the logical features of the interface between the software and hardware components, including the equipment supported and how the equipment and protocol is supported. 

Defines the interfaces according to the content and format of the software/hardware protocol. If the interfaces have been clearly described in other documents, it is not necessary to describe in detail here. But the reference of those documents should be given.

在此描述软件产品和系统硬件组件之间接口的逻辑特征,也包括支持哪些设备、怎样支持这些设备和协议等。
 
按软/硬件协议内容和格式定义接口。如果接口已在其它文档中很清楚地描述,就没有必要在这儿进行详细描述,但需说明应参考的文档。

\subsection{通讯接口}
<This should specify the various interfaces to communications such as local network protocols, etc.>

详细描述通讯接口,如本地网络协议等。

Defines the interfaces according to the content and format of the message/function. If the interfaces have been clearly described in other documents, it is not necessary to describe in detail here. But the reference of those documents should be given.

按消息/函数内容和格式定义接口。如果接口已在其它文档中很清楚地描述,就没有必要在这儿进行详细描述,但需说明应参考的文档。
