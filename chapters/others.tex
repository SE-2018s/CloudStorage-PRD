\chapter{其他需求}

\section{数据库}
    
    1. 系统运行时服务器端的数据库一直在内存中工作

    2. 请求出现时自动建立所需数据表和索引

    3. 允许手动的导入和导出数据

    4. 数据库数据的冗余备份
\section{操作}
\subsection{用户的基本操作}

    1. 注册

    2. 登录

    3. 退出

    4. 找回密码

    5. 更改绑定邮箱

    6. 上传文件或文件夹

    7. 下载文件或文件夹

    8. 新建文件夹

    9. 打开文件夹

    10. 返回上级文件夹

    11. 查看文件属性

    12. 排序显示

    13. 重命名文件或文件夹

    14. 复制、剪切、粘贴文件或文件夹

    15. 创建/关闭标签页
\subsection{用户的特殊操作}

    1. 将文件或文件夹移入回收站

    2. 将文件或文件夹从回收站彻底删除

    3. 将文件或文件夹从回收站移出

    4. 将文件或文件夹加入/移出收藏夹

    5. 加密文件或文件夹
    
    6. 分享文件或文件夹

    7. 对文件或文件夹分类

    8. 搜索文件或文件夹

    9. 预览文件或文件夹

    10. 在线解压/压缩

    11. 举报非法内容

    12. 创建/打开共享文件夹

\subsection{管理员的操作}

    1. 审核用户上传的文件,同时审核举报内容

    2. 管理共享文件夹的权限

    3. 更新版本

    4. 对违规用户封号

    5. 分析磁盘占用、IOPS、网络带宽

    6. 维护并扩展服务器的磁盘、带宽

\section{本地化}

本云盘目前面向的是国内用户,所以必须支持简体中文。另外,为了方便其他用户,以及进一步的推广,需要支持英文版本。语言根据浏览器默认语言设置,用户也可以自行切换。

\section{测试需求}
管理员能够在发现漏洞、异常,以及更新版本之后能够及时测试、试运行。

\section{错误处理}
本云盘的磁盘使用量会比较高,必须考虑各种磁盘损坏等异常;同时本云盘的网络使用量也很大,必须保证网络中断的异常。针对这些错误异常,系统要能够合理的自适应处理行为。

\section{可扩展的易用接口}
本云盘的服务器框架,应该设置充分易用的接口,方便未来的进一步扩展,与第三方开源社区的兼容。

